\documentclass[12pt]{article}


\begin{document}
\tableofcontents


\section{Cystoscopy}
Cystoscopy is endoscopy of the urinary bladder via the urethra. It is carried out with a cystoscope. The urethra is the tube that carries urine from the bladder to the outside of the body. The cystoscope has lenses like a telescope or micro





\subsection{Working}
During a cystoscopy, your doctor: Slides a lubricated cystoscope through the urethra to the bladder. Injects sterile salt water through the cystoscope into the bladder. A stretched, full bladder makes it easier to see the bladder lining.

\subsection{Principle}
Cystoscopy can help find problems with the urinary tract. This may include early signs of cancer, infection, narrowing, blockage, or bleeding.

\subsection{Uses}

Cystoscopy can help find problems with the urinary tract. This may include early signs of cancer, infection, narrowing, blockage, or bleeding. To do this procedure, a long, flexible, lighted tube, called a cystoscope, is put into the urethra and moved up into the bladder.

\subsection{Advantages}
Cystoscopy can help find problems with the urinary tract. This may include early signs of cancer, infection, narrowing, blockage, or bleeding. To do this procedure, a long, flexible, lighted tube, called a cystoscope, is put into the urethra and moved up into the bladder.


\subsection{Disadvantages}
Cystoscopy can help find problems with the urinary tract. This may include early signs of cancer, infection, narrowing, blockage, or bleeding. To do this procedure, a long, flexible, lighted tube, called a cystoscope, is put into the urethra and moved up into the bladder.

\section{Apheresis}
Apheresis units incorporate polyvinyl tubing that draws blood
from the patient and moves it through centrifuges and/or filters
to separate blood products. The blood is then returned to the
patient via tubing or is collected in bags, often suspended from
a pole, for donation or disposal. A display and control panel
allow the operator to program the unit and view progress and/
or alerts. Safety features include pressure sensors, ultrasonic
air-bubble detectors, optical fluid-level detectors, and dry-heat
fluid warmers. The warmers help prevent hypothermia caused
by infusing low-temperature fluids. The unit may have wheels or
it may be placed on a cart.





\subsection{Working}
Apheresis machines are wheeled to the bedside or donor
chair.
•	 Operator connects the sterile tubing sets, also called
pheresis sets, to the patient or donor.
•	 As the patient’s blood is pumped into the machine, an
anticoagulant is automatically added and the blood enters
the chamber.
•	 Blood components are separated using centrifugation and/
or filtration; the method of separation depends on the
product that is to be removed or collected from the blood
\subsection{Principle}
The operator programs the unit to run the desired separation
protocol; the unit automatically controls the centrifuge, pump,
and other settings. Rotary peristaltic pumps draw blood from the
patient or donor and pump blood through filters or centrifuges.
Filters separate blood components based on size; centrifuges
separate by density. Optical sensors detect plasma-cell
interfaces to minimize contamination from other components.
Centrifuges have inlet and outlet ports and compartments to
keep components separated. Pumps move the blood products
into collection bags, add anticoagulant, and reinfuse fluids.
Replacement fluids (e.g., saline, serum albumin, plasma protein
fraction, fresh frozen plasma) are infused into the patient to
maintain appropriate intravascular volume and pressure.

\subsection{Uses}

Apheresis units incorporate polyvinyl tubing that draws blood from the patient and moves it through centrifuges and/or filters to separate blood products. The blood is then returned to the patient via tubing or is collected in bags, often suspended from a pole, for donation or disposal.

\subsection{Advantages}

Automated (apheresis) component collection has the advantages of controlled volumes or doses of component, efficient use of the donor, multiple components from the same donor, better inventory control, and better quality control due to less manipulation of the individual components.

\subsection{Disadvantages}

Air embolism is a risk despite the presence of air-bubble detectors
in most apheresis units. Complications related to double-lumen
venous catheter placement, such as vascular erosion and
perforation, have been reported. Hemolysis
is associated with kinked tubing or poorly
connected sets. Decreased ionized calcium in
the plasma could lead to cardiac arrhythmia.
Blood or blood component transfusions pose
a risk of infection such as hepatitis and AIDS.
\clearpage
\section{sphygmomanometer}

A sphygmomanometer is a device that measures blood pressure. It is composes of an inflatable rubber cuff, which is wrapped around the arm. A measuring device indicates the cuff's pressure. A bulb inflates the cuff and a valve releases pressure. A stethoscope is used to listen to arterial blood flow sounds.

As the heart beats, blood forced through the arteries cause a rise in pressure, called systolic pressure, followed by a decrease in pressure as the heart's ventricles prepare for another beat. This low pressure is called the diastolic pressure



\subsection{Working}
To begin blood pressure measurement, use a properly sized blood pressure cuff. The length of the cuff's bladder should be at least equal to 80% of the circumference of the upper arm.
Wrap the cuff around the upper arm with the cuff's lower edge one inch above the antecubital fossa.
Lightly press the stethoscope's bell over the brachial artery just below the cuff's edge. Some health care workers have difficulty using the bell in the antecubital fossa, so we suggest using the bell or the diaphragm to measure the blood pressure.
Rapidly inflate the cuff to 180mmHg. Release air from the cuff at a moderate rate (3mm/sec).
Listen with the stethoscope and simultaneously observe the dial or mercury gauge. The first knocking sound (Korotkoff Sounds) is the subject's systolic pressure. When the knocking sound disappears, that is the diastolic pressure (such as 120/80).
Record the pressure in both arms and note the difference; also record the subject's position (supine), which arm was used, and the cuff size (small, standard or large adult cuff).
If the subject's pressure is elevated, measure blood pressure two additional times, waiting a few minutes between measurements.
A BLOOD PRESSURE OF 180/120mmHg OR MORE REQUIRES IMMEDIATE ATTENTION!

\subsection{Principle}
The sphygmomanometer cuff is inflated to well above expected systolic pressure. As the valve is opened, cuff pressure (slowly) decreases. When the cuff’s pressure equals the arterial systolic pressure, blood begins to flow past the cuff, creating blood flow turbulence and audible sounds. Using a stethoscope, these sounds are heard and the cuff’s pressure is recorded. The blood flow sounds will continue until the cuff’s pressure falls below the arterial diastolic pressure. The pressure when the blood flow sounds stop indicates the diastolic pressure.

\subsection{Uses}

sphygmomanometer, instrument for measuring blood pressure. It consists of an inflatable rubber cuff, which is wrapped around the upper arm and is connected to an apparatus that records pressure, usually in terms of the height of a column of mercury or on a dial (an aneroid manometer).

\subsection{Advantages}
The biggest advantage of using mercury sphygmomanometers is that they are quite easy to use, and if used properly, can last a lifetime. The device can produce the most accurate results without requiring much readjustment.



\subsection{Disadvantages}

High chances of observer error.
Noise interference can result in inaccurate readings.
Small movements can cause inaccurate readings.
Difficult to monitor in infants or small children.
Mercury can cause an environmental hazard.
Aneroid loses its accuracy over time.
\clearpage

\section{Spirometry}
Spirometry (spy-ROM-uh-tree) is a common office test used to assess how well your lungs work by measuring how much air you inhale, how much you exhale and how quickly you exhale. Spirometry is used to diagnose asthma, chronic obstructive pulmonary disease (COPD) and other conditions that affect breathing





\subsection{Working}
It has a mouthpiece that looks like a vacuum tube. When you inhale with it, the suction will move a disc or a piston up inside a clear cylinder. The deeper you breathe, the higher the piston rises. Most spirometers have numbers on the cylinder to show how much air you take in.

\subsection{Principle}
The test works by measuring airflow into and out of your lungs. To take a spirometry test, you sit and breathe into a small machine called a spirometer. This medical device records the amount of air you breathe in and out as well as the speed of your breath

\subsection{Uses}
Spirometry is used to diagnose asthma, chronic obstructive pulmonary disease (COPD) and other conditions that affect breathing. Spirometry may also be used periodically to monitor your lung condition and check whether a treatment for a chronic lung condition is helping you breathe better
Your doctor may suggest a spirometry test if he or she suspects your signs or symptoms may be caused by a chronic lung condition such as:

Asthma
COPD
Chronic bronchitis
Emphysema
Pulmonary fibrosis

\subsection{Advantages}

A spirometer can be used for a number of reasons such as for testing of Pulmonary Function Tests
Many lung diseases can be ruled out via a Spirometer.
They can assess the effect of the medication and whether or not the medication needs to be adjusted
They measure the progress of the disease treatment
They are optimal for checking the lung function before someone undergoes surgery
They have the ability to measure if chemicals in work environment affect lung function
Can help find the cause of shortness of breath and other lung-related conditions, diseases, and problems


\subsection{Disadvantages}

In general, there are very few risks or possible complications with regular incentive spirometer usage, but it's important to stop if you find yourself becoming lightheaded.

There are rare reports of collapsed lung (pneumothorax) that have been associated with very aggressive spirometry in people with emphysema. If any of the following apply, you shouldn't use an incentive spirometer:3

You've recently had eye surgery: The pressure of breathing forcefully may affect your eyes.
You have a collapsed lung
You have an aneurysm (ballooning blood vessel) in the chest, abdomen, or brain
\clearpage
\section{Laser Tattoo Remover}

Lser tattoo remover removes the tattoo which is engraved on your skin.



\subsection{Working}
Laser beams use concentrated bursts of energy to heat up the ink beneath the skin, which breaks the ink into smaller particles. Tattoos with different colors might require the use of multiple lasers operating at different frequencies. Those smaller ink particles can then be eliminated naturally by the body’s own immune system.

\subsection{Principle}

Unlike a laser pointer that produces a continuous beam of light, tattoo removal lasers produce pulses of light energy. Each pulse of energy penetrates the skin and is absorbed by the tattoo ink. As the tattoo ink particles absorb the energy, they heat up and then shatter into tiny fragments.
\subsection{Uses}

laser tattoo removal is the safest and most effective method for getting rid of that old ink available today. Because it uses only lasers, it is a relatively noninvasive treatment that targets only the ink of your unwanted tattoo and leaves the surrounding skin unaffected.

\subsection{Advantages}
No scarring – The laser light is designed to leave your healthy skin cells alone. This means that the potential for scarring is low, which contrasts removal methods that rely upon “sanding” the skin.
Effective fading and removal of tattoos – Laser removal can effectively diminish the appearance of tattoos without the lasting effects that can make other methods of removal undesirable.
Minimal recovery – After receiving treatments of laser tattoo removal, you need to protect your skin from ultraviolet radiation for a few days. A small amount of redness and tenderness may persist, but these effects should subside within a few days of treatment.
Removing specific or entire tattoos – Laser tattoo removals can remove specific tattoos or cleanse entire regions of your body from tattoos.
Safety – Laser tattoo removal is one of the safest ways to remove tattoos. The risk of infection is minimal, few undesirable side effects occur after receiving treatment and comfort tends to stay high during the treatment process.


\subsection{Disadvantages}

Painful. Many patients believe tattoo removal is more painful than actually getting the tattoo. Besides pain, blistering, crusting, pinpoint hemorrhage, and urticarial reaction can occur. Blistering is a natural part of the healing process of laser tattoo removal.  As the laser breaks up the ink in your skin, it can also break tiny blood vessels around the tattoo resulting in blisters, which usually take up to two weeks to heal.
Risk of Infection. After the procedure, the skin becomes inflamed which increases the chance of infection.
Uneven Skin Colour. Laser tattoo removal can cause hypopigmentation where the treated skin can become lighter than the surrounding skin or can also cause hyperpigmentation where the treated skin becomes darker than the surrounding skin. Dark skinned patients are more prone to uneven skin colour from laser tattoo removal.
Many treatments required. On average 10 treatments are required to remove a tattoo.  With a 6 week between treatments, it would take 1 to 2 years to complete the tattoo removal process. Complete disappearance is not always achieved even after performing numerous laser sessions.  Unfortunately, the number of sessions isn’t something that can be easily predetermined during an initial consultation.
Expensive.  Laser tattoo removal is expensive because the machine is expensive which needs to be factored into the price per treatment.  Plus, due to the number of treatments required, a complete removal can cost thousands of dollars.





\end{document}